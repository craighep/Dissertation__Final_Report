\chapter{Background \& Objectives}
Before commencing the design of the application and the project planning, it is important to have analysed what I hope to have achieved at the end of my project time, and also what steps I will need to be taking to implement each feature. As I will mention later, using FDD will play a big part of how I shape my project and create each feature whether it be by priority, size or difficulty. This section details my understanding for my project requirements, steps I am going to need to take, and also the first sections of an FDD project; developing an overall model and building the list of requirements and features.

\section{Background}
The choice of undertaking a project such as this one was due to two combining factors: maths and an interest to learn graphical programming. The fact that this application will require me to learn graphics, and how to implement visual effects representing the requirements of the project in ways completely new to myself. As graphics is something I have not had much to do with in the past, this project appears both exciting and daunting task due to the learning curve I will need to take. As for the maths factor, I can assume quite a lot of maths will be involved(especially for creating curves, rolls and turns along most of the manoeuvres) which I enjoy learning about.

In terms of the history of the topic, OLAN was originally developed by Michael Gorden in \textbf{2006} and was designed to provide shorthand notation for pilots planning out aerobatic routines without having to draw out the full Aresti diagrams. In recent years, the OLAN notation became used much more until because of licensing issues with the original owner was taken offline. Because of this, a new form of the notation has been created in a more open source way paving the way for applications such as this project's intended aim. Although in this report and my planned application itself will be still referring the notation as OLAN, the new re-make of the language is known as the \textbf{'OpenAero language'}. This is based off of the original, yet is open and allows anyone to use it. In combination with this, the creators of the OpenAero language also developed a web-based \textbf{application} that allows the conversion of the notations to 2D Aresti diagrams. This is somewhat similar to what I hope to achieve, but alongside plenty more features most importantly the ability to see a plane perform the moves.

As for Aresti, named after its conceiver José Luis Aresti Aguirre \textbf{https://www.aerobatics.org.uk/aresti} is the diagram format that OLAN achieves, and represent informative diagrams showing the shape of the routine, direction of travel, rolls and sharpness of turns. Aresti diagrams also can include angles or turns, ranging from 90 degrees to 270. Each diagram usually has a name, relating normally to the shape of the manoeuvre, though some are more commonly known to pilots rather than the regular user. The OLAN notation for each diagram usually attempts to try describe the manoeuvre with the letter used, such as 'o' for a loop, or 'z' for a shark tooth. The full list of manoeuvres, including their OLAN notation and full name can be found on the \textbf{OpenAero site}.

Upon starting this project, several meetings with the project supervisor are planned each providing more detail of the initial requirements. Each of the meetings will be found in the Gantt for the project, with a smaller document in the\textbf{ appendix} showing the results of each meeting. Because I have already attended several meetings at this stage, I can provide a fairly accurate list of initial requirements of the application.\\

\noindent The initial required application can be broken down into an extensive(but less detailed) list of main functional requirements. These are as follows:
\begin{enumerate}
	\item Provide a web-implemented tool that allows input of the OLAN 1 None-IE due to WebGL capabilities. Will it use a simple JSON file to store notations? characters as a string format, alongside possible click functionality.
	\item Relate each notation or set of notations to a certain procedural movement(rotations, movements etc.). 
		\begin{itemize}
			\item Must consider parameters in some of the notations, such as the speed of entry.
		\end{itemize}
	\item Provide a means of linking up these movements in such a way into moves, or the angle of the plane. They should produce a fluid manoeuvre.
	\item Display this using WebGL. Libraries to consider that could help. Begin by initially testing simple shapes to move and fly around, then add textures, and plane structure.
	\item Are libraries OK to use? with some of the movements:
		\begin{itemize}
			\item glMatrix- JavaScript library for helping with performing actions to matrices- http://glmatrix.net
			\item ThreeJS- Another JavaScript library, good with handling cameras and different views- http://threejs.org
		\end{itemize}
	\item Allow user to add different effects such as wind, gravity changes and other physics. Could be better to implement these last, as it will be easier to test pure functionality of rolls etc. first, then figure out natural physics.
	\item Add functionalities of different viewpoints(on-board views, side views) to application.
	\item Possibility to add function to save (using local storage?) users different sets of manoeuvres?
\end{enumerate}

The list shown above also has an \textbf{accompanying} report which I created after my first project meeting. This report includes the list of initial requirements, alongside footnotes, and also a detailing of the methodology and process I plan to follow. The document can be found in the appendices section of this report. 

\section{Analysis}
Upon 
%Taking into account the problem and what you learned from the background work, what was your analysis of the problem? How did your analysis help to decompose the problem into the main tasks that you would undertake? Were there alternative approaches? Why did you choose one approach compared to the alternatives? There should be a clear statement of the objectives of the work, which you will evaluate at the end of the work. In most cases, the agreed objectives or requirements will be the result of a compromise between what would ideally have been produced and what was felt to be possible in the time available. A discussion of the process of arriving at the final list is usually appropriate.

\section{Process}
%You need to describe briefly the life cycle model or research method that you used. You do not need to write about all of the different process models that you are aware of. Focus on the process model that you have used. It is possible that you needed to adapt an existing process model to suit your project; clearly identify what you used and how you adapted it for your needs.

