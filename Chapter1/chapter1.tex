\chapter{Background \& Objectives}
Before commencing the design of the application and the project planning, it is important to have analysed what I hope to have achieved at the end of my project time, and also what steps I will need to be taking to implement each feature. As I will mention later, using FDD will play a big part of how I shape my project and create each feature whether it be by priority, size or difficulty. This section details my understanding for my project requirements, steps I am going to need to take, and also the first sections of an FDD project; developing an overall model and building the list of requirements and features.

\section{Background}
The choice of undertaking a project such as this one 

\section{Analysis}
%Taking into account the problem and what you learned from the background work, what was your analysis of the problem? How did your analysis help to decompose the problem into the main tasks that you would undertake? Were there alternative approaches? Why did you choose one approach compared to the alternatives? There should be a clear statement of the objectives of the work, which you will evaluate at the end of the work. In most cases, the agreed objectives or requirements will be the result of a compromise between what would ideally have been produced and what was felt to be possible in the time available. A discussion of the process of arriving at the final list is usually appropriate.

\section{Process}
%You need to describe briefly the life cycle model or research method that you used. You do not need to write about all of the different process models that you are aware of. Focus on the process model that you have used. It is possible that you needed to adapt an existing process model to suit your project; clearly identify what you used and how you adapted it for your needs.

