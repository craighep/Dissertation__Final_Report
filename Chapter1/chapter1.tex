\chapter{Background \& Objectives}
Before commencing the design of the application and the project planning, it is important to have analysed what I hope to have achieved at the end of my project time, and also what steps I will need to be taking to implement each feature. As I will mention later, using FDD will play a big part of how I shape my project and create each feature whether it be by priority, size or difficulty. This section details my understanding for my project requirements, steps I am going to need to take, and also the first sections of an FDD project; developing an overall model and building the list of requirements and features.

\section{Background}
The choice of undertaking a project such as this one was due to two combining factors: maths and an interest to learn graphical programming. The fact that this application will require me to learn graphics, and how to implement visual effects representing the requirements of the project in ways completely new to myself. As graphics is something I have not had much to do with in the past, this project appears both exciting and daunting task due to the learning curve I will need to take. As for the maths factor, I can assume quite a lot of maths will be involved(especially for creating curves, rolls and turns along most of the manoeuvres) which I enjoy learning about.\\   \\ \\
In terms of the background of OLAN and Aresti\\ \\ \\
Upon starting this project, several meetings with the project supervisor are planned each providing more detail of the initial requirements. Each of the meetings will be found in the Gantt for the project, with a smaller document in the\textbf{ appendix} showing the results of each meeting. Because I have already attended several meetings at this stage, I can provide a fairly accurate list of initial requirements of the application.\\

\noindent The initial required application can be broken down into an extensive(but less detailed) list of main functional requirements. These are as follows:
\begin{enumerate}
	\item Provide a web-implemented tool that: 1 that allows input of the OLAN 1 None-IE due to WebGL capabilities. Will it use a simple JSON file to store notations? characters as a string format, alongside possible click functionality.
	\item Relate each notation or set of notations to a certain procedural
movement2 (rotations, movements etc.). 2 Must consider parameters in some of
the notations, such as the speed of entry
	\item Provide a means of linking up these movements in such a way into moves, or the angle of the plane.
they produce a fluid manoeuvre.
	\item Display this using WebGL3. Libraries4 to consider that could help 3 Begin by initially testing simple shapes
to move and fly around, then add
textures, and plane strcture.
	\item Are libraries ok to use?
with some of the movements:
• glMatrix- Javascript library for helping with performing actions
to matrices- http://glmatrix.net
• ThreeJS- Another Javascript library, good with handling cameras
and different views- http://threejs.org
	\item Allow user to add different effects such as wind, gravity changes
and other physics5. 5 Could be better to implement these
last, as it will be easier to test pure
functionality of rolls etc first, then
figure out natural physics.
	\item Add functionalities of different viewpoints(on-board views, side
views) to application.
	\item Possibility to add function to save (using local storage?) users
different sets of manoeuvres?
\end{enumerate}
%Taking into account the problem and what you learned from the background work, what was your analysis of the problem? How did your analysis help to decompose the problem into the main tasks that you would undertake? Were there alternative approaches? Why did you choose one approach compared to the alternatives? There should be a clear statement of the objectives of the work, which you will evaluate at the end of the work. In most cases, the agreed objectives or requirements will be the result of a compromise between what would ideally have been produced and what was felt to be possible in the time available. A discussion of the process of arriving at the final list is usually appropriate.

\section{Analysis}


\section{Process}
%You need to describe briefly the life cycle model or research method that you used. You do not need to write about all of the different process models that you are aware of. Focus on the process model that you have used. It is possible that you needed to adapt an existing process model to suit your project; clearly identify what you used and how you adapted it for your needs.

