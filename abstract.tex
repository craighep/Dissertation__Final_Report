\thispagestyle{empty}

\begin{center}
{\LARGE\bf Abstract \& Background}
\end{center}

The aim of this project is to create a 3D representation of aerobatic manoeuvres, primarily using the OLAN notation(or one letter aerobatic notation). Both real-scale and remote control aerobatic planes and helicopters use the notation to describe a set of manoeuvres in an overall routine, usually including translating the notations into Aresti symbols. The Aresti symbols came before OLAN, but OLAN was developed to make it possible for pilots to write down quickly their planned routines.\\
The primary element of this project will involve allowing users to insert their notated routines into the application(via an input box on a web-page) thus producing first the ribbon shapes of the manoeuvres, followed by the ability to see a craft fly the route. Although there is only a finite amount of manoeuvres possible from the Aresti catalogue, each OLAN notation can have its own parameters. This can range from entry length into a loop or turn, to the number of rolls in a section of a manoeuvre. The application will also need to be taking into account flight speeds, and possibly wind and gravitational effects. Due to a long list of possible features, the application will be created in a feature-by-feature means (FDD).\\
WebGL will be the language used to create this application, with hopefully object-orientated Javascript to power the application. Libraries helping towards the graphical/ visual side of the application may be required to give greater flexibility and better aesthetic value. All considerations will be found in the analysis and design stages of this report.