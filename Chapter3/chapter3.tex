\chapter{Implementation \& Testing}
Following from the analysis and design of the application, it is now possible to start implementing the features produced from investigating the requirements earlier in the project. As it was already stated that the project would be running under an FDD means, rather than having two sections (one for implementation and one for testing), this section will show each feature one-by-one. By following this scheme, the feature prioirty list will be utalised to ensure the more important features are done first, and that they fulfil their function before moving on. However, this should not hide the fact that issues are possible when implementing and any such issues could hamper further development. If this does occur, the issues and reasons will be explained, followed by what judgement was made to continue on with development of the application as a whole.

\section{FDD project approach}
As mentioned, the feature driven development methodolodgy will now shape the approach that is taken to create each feature. Although design has already been completed for all features, there are still elements to each iteration that should be outlined before proceeding. The best way to plan this is to give a brief list of steps that will be performed in each iteration. This is as follows:

\begin{enumerate}
\item Give a re-cap on the requirements of the feature
\item Provide a walkthrough of implementation (with code samples)
\item Show tests used (JSUnit, usability tests) and results
\item Describe and differences with the feature and original design
\item If there are any issues, describe and give explanations
\item Give a progress log on the item, and provide dates and length of completion for use in accordance with the Gantt chart originally constructed
\end{enumerate}

It should also be noted that once the duration of the time in the project for the implemtation and testing stage is complete, a burn down and progress report will be possible to generate and will be done so. This will give a good idea of exactly the status of the project (with a calculated percentage), to see how much work was done compared to estimates, and to find out how succesful the project was overall. 

\subsection{Iterative Implimentation}
Before the iterations begin, it should be mentioned that although the order of features to be implemented were decided in the analysis, in order to impliment any back end things and allow for proper testing (usability testing), some basic GUI must be created first. Becuase of this, this will be the first iteration, followed by creation of the JSON manoeuvres. Following these iterations, it will then be much easier to test future features created. Without a basic GUI and canvas, it will be impossible to see the effects created whilst adding features such as drawing flight paths or moving cameras. After these two moved features are complete, the prioritised list will run as planned initially.

\subsubsection{Feature 1- Create a basic GUI}



\section{Final status and progress}
\subsection{Implementation and testing review}
%The implementation should look at any issues you encountered as you tried to implement your design. During the work, you might have found that elements of your design were unnecessary or overly complex; perhaps third party libraries were available that simplified some of the functions that you intended to implement. If things were easier in some areas, then how did you adapt your project to take account of your findings?It is more likely that things were more complex than you first thought. In particular, were there any problems or difficulties that you found during implementation that you had to address? Did such problems simply delay you or were they more significant? You can conclude this section by reviewing the end of the implementation stage against the planned requirements. 

%Detailed descriptions of every test case are definitely not what is required here. What is important is to show that you adopted a sensible strategy that was, in principle, capable of testing the system adequately even if you did not have the time to test the system fully.Have you tested your system on �real users�? For example, if your system is supposed to solve a problem for a business, then it would be appropriate to present your approach to involve the users in the testing process and to record the results that you obtained. Depending on the level of detail, it is likely that you would put any detailed results in an appendix.The following sections indicate some areas you might include. Other sections may be more appropriate to your project. 

%\section{Overall Approach to Testing}
%\section{Automated Testing}
%\subsection{Unit Tests}
%\subsection{User Interface Testing}
%\subsection{Stress Testing}
%\subsection{Other types of testing}
%\section{Integration Testing}
%\section{User Testing}